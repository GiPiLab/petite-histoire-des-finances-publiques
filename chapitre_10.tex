\chapter{En guise de conclusion : budget et loi de finances}

\minitoc

Pour ce qui concerne l'\'{E}tat, la terminologie \og{}budget\fg{} est
remplac�e, dans la Constitution de 1958 par celle de \og{}loi de finances\fg{}.

L'article 1\ier{} de l'ordonnance du 2 juillet 1959 stipule que \og{}les lois de
finances d�terminent la nature, le montant et l'affectation des ressources et
des charges de l'\'{E}tat, compte tenu d'un �quilibre �conomique et financier
qu'elles d�finissent\fg{}.

L'article 16 du m�me texte pr�cise que le budget est constitu� par
\og{}l'ensemble des comptes qui d�crivent, pour une ann�e civile, toutes les
ressources et toutes les charges permanentes de l'\'{E}tat\fg{}.

Il en r�sulte l'�mergence de trois notions diff�rentes, faisant l'objet de
fr�quentes confusions, tant il est courant d'�crire sur le budget de
l'\'{E}tat, alors que ce terme ne d�signe plus que l'ensemble des documents
budg�taires.

Lorsqu'il est souhait� mettre l'accent sur les notions de pr�vision ou
d'autorisation, il convient d'utiliser les termes \og{}loi de finances\fg{} ou
\og{}budget g�n�ral\fg{}.

La loi de finances est le texte l�gislatif vot�, chaque ann�e, par le
Parlement. Elle regroupe l'int�gralit� des charges et des ressources de
l'\'{E}tat. Elle est modifi�e, en cours d'ann�e, par la loi de finances dite
\og{}rectificative\fg{} et apr�s la fin de l'exercice, par la loi de r�glement.

Le budget g�n�ral s'ins�re dans la loi de finances dont il ne regroupe que les
d�penses de l'\'{E}tat � caract�re d�finitif, financ�es essentiellement par des
recettes fiscales.

Les recettes et les charges � caract�re temporaire sont group�es sous le
vocable de \og{}comptes sp�ciaux du Tr�sor\fg{}.


\begin{center}
\emph{Hormis ses annexes, cette cr�ation est mise � disposition selon le Contrat 
Paternit�-NonCommercial-ShareAlike 2.0 France disponible en ligne :}

\url{http://creativecommons.org/licenses/by-nc-sa/2.0/fr/}

\emph{ou par courrier postal � :}

\texttt{Creative Commons, 559 Nathan Abbott Way, Stanford, California 94305, 
USA.}

\end{center}
